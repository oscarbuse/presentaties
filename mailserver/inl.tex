\section{Inleiding}

\begin{frame}
	\frametitle{Inleiding}
Een eigen mailserver is leuk ende leerzaam maar:
\begin{itemize}
\pause
\item {\it ''Great power gives great responsibility.''}
\pause
\item Belangrijk dat je eigen mailserver door andere mailservers serieus genomen wordt.\\
\pause
\item Hier enkele tips.
\end{itemize}
\end{frame}

\section{Enkele tips}
\subsection{spam?}
\begin{frame}
	\frametitle{Waarom wordt mijn mail als spam gezien?}
\begin{itemize}
\pause
\item Je mailserver IP-adres is door de ontvanger op een blacklist gezet (..)
	\pause
	\begin{itemize}
	  \item Nodig je actief af te melden
	\end{itemize}
\pause
\item Je reverse DNS ''match niet''
	\pause
	\begin{itemize}
	  \item Zorg dat DNS <=> Reverse DNS
	  \pause
      \item {\tt
      mail.all-stars.nl -> 149.210.136.182
      149.210.136.182   -> mail.all-stars.nl}
	\end{itemize}
\pause
\item Je ''groet'' niet volgende de RFC. Zorg voor een juiste ''groet'' (HELO).
	\pause
	\begin{itemize}
	\item In /etc/postfix/main.cf:\\
    {\tt	myhostname = mail.all-stars.nl
			smtpd\_banner = \$myhostname}
	\end{itemize}
\end{itemize}
\end{frame}

\begin{frame}
	\frametitle{Waarom wordt mijn mail als spam gezien?}
\begin{itemize}
\item Een ontvangende mailserver constateert dat er geen (of geen goed) {\it SPF-record} is.\\
\pause
SPF (Sender Policy Framework): een {\bf ontvangende} mailserver checkt of de {\bf afzender} (bv. oscar{\myat}kwalinux.nl) via de {\bf gekozen} SMTP-server mag versturen.
    \begin{itemize}
	\pause
    \item In DNS voor kwalinux.nl is een SPF-record nodig:
    {\tt kwalinux.nl.            599     IN      TXT     ''v=spf1 mx -all''}
	\end{itemize}
\pause
~\\
SPF checkt alleen de MAIL FROM (= envelope sender).\\
''Sender ID'' (spf2..) kan ook de header checken.
\end{itemize}
\end{frame}

\begin{frame}
	\frametitle{Waarom wordt mijn mail als spam gezien?}
\begin{itemize}
\item Ontvangende mailserver constateert dat je mail niet gesigned is.\\
\pause
Met gesignde mail kan de ontvangende mailserver verifieren dat mail echt van het gebruikte domein (in het voorbeeld kwalinux.nl) afkomstig is. 
	\pause
	\begin{itemize}
    \item Install en configureer opendkim (zie bv. http://tecadmin.net/setup-dkim-with-postfix-on-ubuntu-debian/)
	\pause
    \item In DNS voor bv. kwalinux.nl:\\
	{\tiny
	{\tt mail.\_domainkey.kwalinux.nl. 86400 IN   TXT     ''v=DKIM1\; k=rsa\; t=y\;
p=MIGfMA0GCSqGSIb3DQEBAQUAA4GNADCBiQKBgQDJoWCzfnp32KftMreHN0mQkXc0ZQp4M+
StkCSzPEJr5REq9/1wOjktO9R1NNPsU9SV0MtK2i8gbXMDbQJUQCe+vtJlcThGPatYMO8nIb
BXkJ3Lcw6b6+Dmj6NwXPyDDAcUxObmggnfl9hXr2JG8Fhmi+x21LyAOCrwwNtw9T/opQIDAQAB''}}
	\end{itemize}
\end{itemize}
SPF en DKIM: nog zwakheden (o.a. SPF met Forwarding, DKIM met ''double From:'' en beide geen feedback naar sender).\\
\pause
DMARC?
\end{frame}

\section{Samengevat}
\begin{frame}
    \frametitle{Samengevat}
Samengevat de genoemde maatregelen:
\begin{itemize}
\pause
\item zorg dat je niet op een blacklist staat. Zo ja, probeer je af te melden.
\pause
\item zorg voor goede reverse DNS.
\pause
\item zorg voor een goede ''groet'' (HELO).
\pause
\item maak een SPF-record voor je versturende domein.
\pause
\item sign uitgaande mail (DKIM).
\pause
\item DMARC?
\end{itemize}
\end{frame}

\section{Conclusie}
\begin{frame}
    \frametitle{Conclusie}
\begin{itemize}
\pause
\item Een eigen mailserver is leuk, flexibel maar:\\
\pause
	\begin{itemize}
	\item Zorg voor een goede en up-2-date configuratie.\\
	\pause
	\item Blijf voldoen aan de anti-spam maatregelen die bv. de mailservers van gmail, yahoo en microsoft implementeren. 
	\pause
	\item Blijf op de hoogte van nieuwe ontwikkelingen.
	\end{itemize}
\pause
\item Zorg dat je ook zelf je DNS kan updaten.
\pause
\item Test je setup!
	\begin{itemize}
    \pause
	\item Bv. door vanaf je account een mail te sturen naar test{\myat}allaboutspam.com\\
	Voorbeeld uitkomst: \url{www.all-stars.nl/mx.html}\\
    \pause
	\item Google op ''mailserver check''.\\
		  Check bv. via \url{http://mail-server-test.online-domain-tools.com/}
	\end{itemize}
\end{itemize}
\end{frame}

\section{Vragen?}
\begin{frame}
    \frametitle{Vragen?}
\end{frame}
