%%%%%%%%%%%%%%%%%%%%%%%%%%%%%%%%%%%%%%%%%%%%%%%%%%%%%%%%5
% basis specials
%%%%%%%%%%%%%%%%%%%%%%%%%%%%%%%%%%%%%%%%%%%%%%%%%%%%%%%%5
% make "impressive" work (only with pdf version 4..)
\pdfminorversion=4

% choose between handout format and beamer format.
%%\documentclass[t,12pt,handout,xcolor={dvipsnames,table}]{beamer}
%\documentclass[t,10pt,handout,xcolor=table]{beamer}
\documentclass[t,xcolor=table]{beamer}

%%%%%%%%%% styling
\usepackage{../presentaties}
\usepackage{../frames}

%%%%%%%%%%%%% Text placement
%\setbeamerfont{block title}{size={}}
\setbeamersize{text margin left=1.5cm,text margin right=1cm}
\setbeamertemplate{frametitle}{\insertframetitle\vskip-11pt\vskip15pt\hfill\usebeamerfont{framesubtitle}\insertframesubtitle\vskip-24pt\hrulefill}
\setbeamertemplate{sections/subsections in toc}[default]

\setbeamerfont{section in toc}{size=\scriptsize}
\setbeamerfont{subsection in toc}{size=\scriptsize}
\renewcommand{\ttdefault}{pcr}

%%%%%%%%%% description list
% align left with our description list
\defbeamertemplate{description item}{align left}{\insertdescriptionitem\hfill}
\setbeamertemplate{description item}[align left]

%%%%%%%%%%% blocks
% round corners for blocks
\setbeamertemplate{blocks}[rounded][shadow=false]
% block title color, fonts in section/subsection in toc
\setbeamercolor{block title}{use=structure,fg=kwalinux}
\setbeamerfont{block title}{size=\small}
% set example block color
%\setbeamercolor*{block body example}{fg= blue, bg= blue!5}

%%%%%%%%%%% special chars, html entities
%\newcommand{\Ent}[1]{\char#1 }
% for |, >, <
\usepackage[T1]{fontenc}
% \ding{nummer}, bv 122 = zwarte cursor
\usepackage{pifont}
% more specials: \Enter etc..
\usepackage{keystroke}
% \frown (not working?)
%\DeclareMathAlphabet\mathbb{U}{msb}{m}{n}
%\usepackage{MnSymbol}
% smilies
\usepackage{wasysym}

% for Verbatim / terminal
\usepackage{fancyvrb,listings}
%\lstdefinestyle{terminal}{basicstyle=\ttfamily}
\newenvironment{terminal}[1][]
  { \VerbatimEnvironment%
    \begin{Verbatim}[#1]}
  { \end{Verbatim}  }

% frame with border
%\fvset{frame=single,framesep=1mm,fontfamily=courier,fontsize=\scriptsize,framerule=.2mm,numbersep=1mm,commandchars=\\\{\}}

%%%%%%%%%%%%%%% % for terminal examples
\usepackage{tcolorbox}
\tcbset{%
    noparskip,
    colback=gray!10, %background color of the box
    colframe=gray!40, %color of frame and title background
}

% translate
\usepackage[dutch]{babel}

% textblocks positioning
%\usepackage[absolute,overlay]{textpos}

%%%%%%%%%%%%%%% drawing
\usepackage{tikz}
% extra for trees (filesystem..)
\usepackage{tikz-qtree}
\usetikzlibrary{arrows, shapes, backgrounds}
\tikzstyle{startstop} = [rectangle, rounded corners, minimum width=3cm, minimum height=1cm,draw=black, fill=black!30, text width=3cm]
\tikzstyle{process} = [rectangle, minimum width=0cm, minimum height=0cm, text centered, draw=black, fill=white, text width=1cm]
\tikzstyle{button} = [minimum width=0cm, minimum height=0cm, text centered, fill=white, text width=1cm]
\tikzstyle{every node}=[font=\small]
%\tikzstyle{arrow} = [thick,->,>=stealth]
% apply the 'remember picture' style. To avoid some typing, we'll apply
% the style to all pictures.
% this is to point correctly from a -> b (with nodes and arrows)
\tikzstyle{every picture}+=[remember picture]
% By default all math in TikZ nodes are set in inline mode. Change this to
% displaystyle so that we don't get small fractions.
\everymath{\displaystyle}


%%%%%%%%%%% more packages
% include pictures
\usepackage{graphicx}
% make clumns
\usepackage{multicol}
% ???
\usepackage{booktabs}
%\usepackage{parskip}
%\setlength{\parskip}{\smallskipamount} 
%%%%%%%%%%%%%%%%%%%%%%%%%%%%%%%%%%%%%%%%%%%%%%%%%%%%%%%%5
\title{DNS}
\subtitle{Linuxnijmegen}
\author{Oscar Buse\\
{\small 13 jan 2015}}
%oscar@kwalinux.nl\\
%http://www.kwalinux.nl
%}
%\subject{LPI 101}
\date{}

% Show an outline of the beginning of a section.
%\AtBeginSection[]
%{
%\setbeamertemplate{footline}[text line]{%
%\parbox{\linewidth}{\vspace*{-8pt}\hfill}}
%\setbeamertemplate{navigation symbols}{}
%\pgfdeclareimage[width=\paperwidth,height=\paperheight]{bg}{blank}
%\usebackgroundtemplate{\pgfuseimage{bg}}
%  \begin{frame}<handout:0>[noframenumbering]
%    \frametitle{Inhoudsopgave}
%    \vspace{-14pt}
%    \fontsize{6}{9}\selectfont
%    \begin{columns}[t]
%        \begin{column}{.5\textwidth}
%                \begin{block}{Inleiding}
%                        \tableofcontents[sections={1-4}]
%                \end{block}
%                \begin{block}{Hoe werkt DNS?}
%                        \tableofcontents[sections={5-10}]
%                \end{block}
%		\begin{block}{De omgekeerde wereld}
%                         \tableofcontents[sections={11}]
%		\end{block}
%                \begin{block}{Zonedata}
%                         \tableofcontents[sections={12}]
%                \end{block}
%        \end{column}
%        \begin{column}{.5\textwidth}
%                \begin{block}{Tools}
%                         \tableofcontents[currentsection,sections={13}]
%                \end{block}
%                \begin{block}{Configuratie van bind}
%                         \tableofcontents[currentsection,sections={14-16}]
%                \end{block}
%                \begin{block}{Gevorderde technieken}
%                         \tableofcontents[currentsection,sections={17}]
%                \end{block}
%                \begin{block}{Q\&A (Dokter DNS)}
%                         \tableofcontents[currentsection,sections={18}]
%                \end{block}
%        \end{column}
%    \end{columns}
%  \end{frame}
%} % /AtBeginSection
%\usebackgroundtemplate%
%{%
%    \includegraphics[width=\paperwidth,height=\paperheight]{img/linux-timeline.png}%
%}
\makeatletter
    \newenvironment{withoutheadline}{
        \setbeamertemplate{headline}[default]
        \def\beamer@entrycode{\vspace*{-\headheight}}
    }{}
\makeatother
%%%%%%%%%%%%%%%%%%%%%%%%%%%%%%%%%%%%%%%%%%%%%%%%%%%%%%%%5
\begin{document}
	\begin{frontframe}
		\titlepage
	\end{frontframe}

	\begin{sectionframe}
    \frametitle{Inhoudsopgave}
\vspace{-14pt}
    \fontsize{6}{9}\selectfont
    \begin{columns}[t]
        \begin{column}{.5\textwidth}
		\begin{block}{Inleiding}
            		\tableofcontents[sections={1}]
		\end{block}
                \begin{block}{Hoe werkt DNS?}
                        \tableofcontents[sections={2}]
                \end{block}
		\begin{block}{De omgekeerde wereld}
           		 \tableofcontents[sections={3}]
		\end{block}
        \end{column}
        \begin{column}{.5\textwidth}
                \begin{block}{Zonedata}
                         \tableofcontents[sections={4}]
                \end{block}
                \begin{block}{Tools}
                         \tableofcontents[sections={5}]
                \end{block}
                \begin{block}{Configuratie van bind}
                         \tableofcontents[sections={6}]
                \end{block}
                \begin{block}{Q\&A (Dokter DNS)}
                         \tableofcontents[sections={7}]
                \end{block}
        \end{column}
    \end{columns}
	\end{sectionframe}

	\section{Inleiding}
\subsection{}
\begin{styleframe}
	\frametitle{Inleiding}
Dit praatje gaat over Elasticsearch.\\
\pause
De onderwerpen die aan bod komen:
\begin{itemize}[<+>]
	\item Wat is Elasticsearch?
	\item Waarom Elasticsearch?
	\item Installatie.
	\item Terminologie.
	\item Wat is je data en hoe wordt deze data opgeslagen?
	\item Werken met Elasticsearch. Theorie met voorbeelden
	\item Een toepassing.
	\item De voor- en nadelen van Elasticsearch.
	\item Referenties.
%	\item Best practices.
\end{itemize}
\end{styleframe}

\section{Wat is ES?}
\subsection{}
\begin{styleframe}
    \frametitle{Wat is Elasticsearch?}
\pause
Definitie: gedistribueerde tekst zoek (en analyse) software met http interface (''RESTful'') en json format.\\
\pause
%%Elasticsearch is document oriented, meaning that it stores entire objects or documents.
%It not only stores them, but also indexes the contents of each document in order to
%make them searchable. In Elasticsearch, you index, search, sort, and filter documents
%—not rows of columnar data. This is a fundamentally different way of thinking about
%data and is one of the reasons Elasticsearch can perform complex full-text search.
Enkele eigenschappen: 
\begin{itemize}[<+>]
%	\item Bijna realtime (\textasciitilde 1 sec vertraging).
%	\pause
	\item Zeer snel (\textbf{alles} wordt default ge-indexeerd).
	\item Geheugen intensief.
	\item ES is document (object) georienteerd (ipv rij/kolom georienteerd).
	\item Goed voor full text search.
	\item Ingebouwde score bij relevantie.
	\item Ingebouwde {\it highlighting}.
	\item Distributed (horizontaal schaalbaar, meerdere nodes).
	\item Ook voor data aggregatie en analyse.
\end{itemize}
\pause
Is allemaal niet nieuw maar het zit wel allemaal samen in 1 applicatie.
\end{styleframe}

\section{Waarom ES?}
\subsection{}
\begin{styleframe}
	\frametitle{Waarom Elasticsearch?}
Zaak om eerst te kijken wat je kunt hebben aan ES.
\begin{itemize}[<+>]
	\item Zie o.a. de eerder genoemde eigenschappen.
	\item Je hebt niks aan data als je er niet wat mee doet.
	\item Leegte in log analyse.. Splunk kan bv. veel maar is duur..
	\item Open Source.
%	\item Cloud vraagt om meer log analyse.
%	\pause
	\item Maken van {\it data-informed} beslissingen.
\end{itemize}
~\\
\pause
Wat kan ES wat bv. niet kan in SQL?\\
\pause
Niet de juiste vraag: gebruik de tool die voldoet voor jouw situatie.\\
\pause
Eenmaal wat meer bekend geef ik een opsomming van de voor- en nadelen.
\end{styleframe}

\section{Installatie}
\subsection{}
\begin{styleframe}
	\frametitle{Installatie}
\begin{itemize}[<+>]
	\item Eenvoudig met downloaden van tgz-file of packages (rpm of deb).
	\item Default luistert Elasticsearch op 127.0.0.1, port 9200.
	\item Ook bij een cluster: 1 interface naar je data.\\
	\item Config in /etc/elasticsearch
	\item OS configuratie, o.a.:
	\begin{itemize}
		\item disable swap (gebruik {\it ES only} servers/VM's). \only<1>{Only 1st slide}
		\item zorg voor genoeg open files (max aantal file descriptors, /proc/security/limits.conf).
		\item zorg voor genoeg mogelijke processen (ulimit -u).
	\end{itemize}
	\item[] Veel wordt al gedaan via de package installers. Check settings met de documentatie (zie referenties).
	\item verschil development en productie omgeving (bv. met setting network.host)
\end{itemize}
\end{styleframe}

\section{Terminologie}
\subsection{}
\begin{styleframe}
	\frametitle{Terminologie}
\pause
\begin{description}[<+>][documentbla]
	\item[Lucene] ES is afgeleid van Apache's Lucene: software voor text indexering en zoeken.
	\item[cluster] een verzameling van servers (nodes) wat elasticsearch mogelijk maakt.
	\item[node] 1 server (of VM) als onderdeel van een cluster.
	\item[index] verzameling data met overeenkomstige kenmerken. Vergelijk met een \textbf{database} in SQL.
	\item[type] verzameling data binnen een index. Vergelijk met een \textbf{table} in SQL.
	\item[field] veld binnen een type. Vergelijk met \textbf{kolom} in SQL.
	\item[document] de basis-eenheid van informatie in ES. Geformatteerd in JavaScript Object Notatie (JSON).\\
\end{description}
\pause
{\scriptsize{\tt\textbf{Relational DB => Databases => Tables => Rows => Columns\\
Elasticsearch => Indices => Types => Documents => Fields}}}
\end{styleframe}


\subsection{}
\begin{styleframe}
\begin{description}[<+>][documentbla]
	\frametitle{}
	\item[term] een unieke ge-indexeerde waarde (''Bla'' is wat anders dan ''bla'')
	\item[shard]
	\begin{itemize}
		\item unit welke een deel van alle data bevat (goed voor opdelen en verdelen van data).
		\item 1 shard kan maximaal 2.147.483.519 {\it documenten} bevatten.
		\item primary en replica shards.
		\item aantal primary shards bepaald hoe groot je db kan zijn.
		\item replica shards zijn readonly. Goed voor redundancy en performance (parallel gebruik van replicas).
		\item default 5 primary en 5 replica shards per index (db).
		\item minimaal 1 shard per node.
		\item aantal primary shards ligt vast, aantal replicas kan gewijzigd worden.
	\end{itemize}
\end{description}
\end{styleframe}

\subsection{}
\begin{styleframe}
	\frametitle{Data opslag}
Hoe wordt de data opgeslagen? 
\begin{itemize}[<+>]
	\item opslag op het filesysteem.
	\item veel gebruik gemaakt van compressie.
	\item data in JSON records.
	\item mapping voor type velden (default: text + keyword (vroeger: string)).
	\item \textbf{Alles} (elk ''field'') wordt ge-indexeerd en is doorzoekbaar.
	\item indexeren met ''inverted'' indexen: (''terms'' naar documents).
	\item verschil in ''full text fields'' (match search) en ''exact value fields'' (term search).
	\begin{itemize}
		\item {\it analyzed} en {\it not analyzed}
		\item Bv. string => text ({\it analyzed}) + keyword (exact, {\it not analyzed})
	\end{itemize}
\end{itemize}
\end{styleframe}

\section{Werken met ES}
\subsection{}
\begin{styleframe}
	\frametitle{Werken met ES}
\pause
Data wordt opgeslagen, opgevraagd, gewijzigd met een REST API (HTTP requests), denk hierbij aan bv.:
\begin{itemize}[<+>]
	\item Document API: voor CRUD operaties.
	\item Cluster API: checken van health, status en verkrijgen van statistieken van je cluster/nodes
	\item Index API: beheer je database (bv. DELETE, CREATE).
	\item Search API: doe (geavanceerde) zoek acties.
\end{itemize}
\pause
Diverse hulpmiddelen voor deze API's: curl, python/perl modules, websites (Kibana), ...\\\
\end{styleframe}

\subsection{}
\begin{styleframefrag}
	\frametitle{Document API}
Deze API ondersteunt diverse CRUD operaties.
\begin{itemize}[<+>]
	\item Url heeft de vorm:\\
{\scriptsize {\tt \textbf{\hspace{1em}POST|PUT|GET|DELETE host:9200/index/type/id -d \{ \} }}}
	\item POST: index (insert met auto-created id)\\
{\scriptsize {\tt \textbf{\hspace{1em}curl -XPOST '127.0.0.1:9200/lugn/mensen'}}}\\
{\scriptsize {\tt \textbf{\hspace{2em}-d '\{''naam'': ''Harry Nak''\}'}}}
	\item PUT: index (insert, geef zelf id op), update (index met zelfde id)\\
{\scriptsize {\tt \textbf{\hspace{1em}curl -XPUT '127.0.0.1:9200/lugn/mensen/1'}}}\\
{\scriptsize {\tt \textbf{\hspace{2em}-d '\{''naam'': ''Harry Nak''\}'}}}
	\item GET: Bv. een hele index (database) doorzoeken kan met:\\
{\scriptsize {\tt \textbf{\hspace{1em}curl -XGET 127.0.0.1:9200/lugn/\_search?q=zoekstring}}}
\end{itemize}
\end{styleframefrag}

\subsection{}
\begin{styleframefrag}
	\frametitle{}
\begin{itemize}[<+>]
	\item DELETE\\
{\scriptsize {\tt \textbf{\hspace{1em}curl -XDELETE 127.0.0.1:9200/lugn?pretty'}}}\\
~\\
{\scriptsize {\tt \textbf{\hspace{1em}curl -XDELETE 127.0.0.1:9200/lugn/mensen/2?pretty'}}}\\
~\\
{\scriptsize {\tt \textbf{\hspace{1em}curl -XPOST}}}\\
{\scriptsize {\tt \textbf{\hspace{2em}127.0.0.1:9200/lugn/mensen/\_delete\_by\_query?pretty \textbackslash}}}\\
{\scriptsize {\tt \textbf{\hspace{2em}-d '\{''query'':\{''match'':\{''naam'':''Harry''\}\}\}'}}}
~\\
	\item UPDATE: (onder de motorkap DELETE en CREATE)\\
{\scriptsize {\tt \textbf{\hspace{1em}curl -XPOST 127.0.0.1:9200/lugn/mensen/1/\_update \textbackslash}}}\\
{\scriptsize {\tt \textbf{\hspace{2em}-d '\{''doc'':\{''naam'':''Harry A. Nak''\}\}'}}}
\end{itemize}
\end{styleframefrag}

% date_typen?
\subsection{}
\begin{styleframefrag}
	\frametitle{}
Zoektechnieken:
%URI searches
query-string search en ''(json) request body'' search. 
\begin{itemize}[<+>]
	\item woord, woo* woor?  
%match_phrase
	\item fieldname:woord (bv. usernaam:harry)
	\item query (match + scores) en filters (match yes or no, geen scores)
	\item match (analyzed, text) en term (exact, keyword)) queries.
	\item aggregate, average, sort
	\item meer .. (Query DSL: flexibele query language in json format. Bv.:
{\scriptsize
\begin{verbatim}
  "query":{
    "filter":{
      "geo_distance":{
        "distance":"100km",
        "location":[32.052098, 76.649294]
      }
    }
  }
\end{verbatim}
}
	\item meer .. (boosts, suggestions, completion)
\end{itemize}
%inverted index: maps ''terms'' -> documents
%de index is opgebouwd uit kleinere bouwstenen : shards (''scherven''). 
%(''normal (forward) index'': documents -> content)
%shards en replicas
%
%As a general rule, use query clauses for full-text search or for any condition that should affect the relevance score, and use filters for everything else.
\end{styleframefrag}

\section{Een ''toepassing''}
\subsection{}
\begin{styleframe}
    \frametitle{Een ''toepassing''}
\pause
\begin{itemize}[<+>]
	\item sql2es.py
	\item search.py
\end{itemize}
\end{styleframe}

\section{Afrondend}
\subsection{}
\begin{styleframe}
    \frametitle{De voor- en nadelen van Elasticsearch.}  
\begin{itemize}[<+>]
	\item - veiligheid? (geen authenticatie).
	\item - geen transactie (commits/rollback) model: minder controle over consistentie van de data.
	\item - relatief nieuw (''moet zich nog bewijzen'').
	\item - query DSL kan erg complex worden (maar geldt bv. ook voor SQL).
	\item + Zeer schaalbaar, snel (alles is een index).
	\item + Lage drempel maar ook zeer flexibele en geavanceerde mogelijkheden.
	\item + veel OOTB: full text search, scoring, aggregatie, suggestie, ''result highlighting''.
\end{itemize}
\end{styleframe}

\begin{styleframe}
    \frametitle{De voor- en nadelen van Elasticsearch.}  
\begin{itemize}[<+>]
	\item + goede integratie met Logstash en Kibana (ELK-stack).
	\item + object storage (ipv key/value).
	\item + veel geolocatie ''features''.
\end{itemize}
% diff met sql:
% - es does do a "relevant" search ("rock climbing"), most relevant search.
% - aggregate more powerful then group by
% - schaalbaarder
% - alles is een index: sneller
% - orientatie op full text search
\pause
Als met alles gebruik de juiste tool voor de juiste toepassing, bv.:
\begin{itemize}[<+>]
	\item betalingstransacties in een RDBMS
	\item zoeken van text in veel documenten: Elasticsearch
\end{itemize}

\end{styleframe}

%\section{Voorbeelden}
%\subsection{}
%\begin{styleframefrag}
%    \frametitle{Voorbeelden}
%\pause
%\begin{description}[documentbla]
%\item[indices] \verb!curl '127.0.0.1:9200/_cat/indices?v'! Overzicht ''databases''
%\item[show ''all''] \verb!curl '127.0.0.1:9200/sw/_search?pretty=true'! Toon records
%\end{description}
%\end{styleframefrag}


%\section{Best practices}
%\subsection{}
%\begin{styleframe}
%    \frametitle{Best practices}
%\pause
%\end{styleframe}

\subsection{}
\begin{styleframe}
	\frametitle{}
Voorbeelden van toepassingen:
\begin{itemize}[<+>]
	\item search engines voor bv. websites.
	\item search en analyseren van allerlei logging (weblog, syslog, klant-data, ...).
	\item wikipedia, the guardian, stack overflow, github, reisavonturen, ...
\end{itemize}
\end{styleframe}


%\section{Referenties}
\subsection{}
\begin{styleframe}
    \frametitle{Referenties}
\begin{itemize}
	\item Website: \url{https://www.elastic.co}
	\item Leren: \url{https://www.elastic.co/learn}
	\item Setup/config: \url{https://www.elastic.co/guide/en/elasticsearch/reference/current/setup.html}
	\item OS settings: \url{https://www.elastic.co/guide/en/elasticsearch/reference/current/system-config.html}
	\item Cheatsheet: \url{http://elasticsearch-cheatsheet.jolicode.com/}
\end{itemize}
\pause
\large Bier?
\end{styleframe}

\end{document}
